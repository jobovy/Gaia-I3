\documentclass[12pt]{article}
\usepackage{amsfonts}
\usepackage{amsmath}
\usepackage{amsthm}
\usepackage{graphicx}
\usepackage{fancyhdr}

\renewcommand{\headrulewidth}{0.4pt}

\addtolength{\oddsidemargin}{-.875in}
\addtolength{\evensidemargin}{-.875in}
\addtolength{\textwidth}{1.75in}
\addtolength{\topmargin}{-.875in}
\addtolength{\textheight}{1.75in}

\begin{document}
\title{Progress Report: (Week of July 3)}
\author{Samuel Wong}
\date{July 3, 2018}
\maketitle

\section{Presentation Preparation}
This week, the team spent significant amount of time preparing for the mid-summer presentation. See the powerpoint in this folder.

\section{Samuel's Accomplished Tasks}
\subsection{Rewrote SampleV on Set on Galpy}
During meeting with Jo last week, I learned that Ayush and I misinterpreted what Jo wanted us to do with sampleV. We thought we should write a function that interpolate the result of sampleV, but that doesn't work. What we really need to do is go into Galpy and change sampleV.

This week I made the necessary changes and it turned out our previous work mostly survived. I first changed the original qdf sample V to take an extra optional argument of max vT. The optimization of this value is what is slowing the program down. I defaulted it to None. In this case, the program optimize it as usual. But if it is given, it uses the given one.

Next, I wrote a sampleV on Set that takes a set of Rz coordinates, generate a grid, and optimize vT on the grid. Then I interpolate the vT at desired positions, and sampleV there using the interpolated vT as the optional argument.

Not only did most of the code survived, the method of KS statistics test also did.
\subsection{Realized the Eigenvalue Method Does Not Work}
Last meeting, after I proposed to find the rank of the matrix $(\nabla \rho, \nabla E, \nabla L_z)$, Jo said the correct answer should be checking how close the eigenvalue of this matrix is to zero. But after thinking very hard about it and discussing it with Mathew, I found out that this concept is wrong since it is a 3 by 6 matrix, and eigenvalue is not defined for non-square matrix.

\section{Ayush’s Accomplished Tasks}
\section{Michael's Accomplished Tasks}
\section{Mathew's Accomplished Tasks}

\end{document}