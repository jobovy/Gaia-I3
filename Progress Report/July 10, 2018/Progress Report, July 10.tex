\documentclass[12pt]{article}
\usepackage{amsfonts}
\usepackage{amsmath}
\usepackage{amsthm}
\usepackage{graphicx}
\usepackage{fancyhdr}

\renewcommand{\headrulewidth}{0.4pt}

\addtolength{\oddsidemargin}{-.875in}
\addtolength{\evensidemargin}{-.875in}
\addtolength{\textwidth}{1.75in}
\addtolength{\topmargin}{-.875in}
\addtolength{\textheight}{1.75in}

\begin{document}
\title{Progress Report: (Week of July 10)}
\author{Samuel Wong}
\date{July 10, 2018}
\maketitle


\section{Samuel's Accomplished Tasks}
\subsection{Fixed a Bug of Changing Point Value in Main}

\subsection{Confirmed Main Program Nan Come from Zero Gradient}

\subsection{Wrote Sample Location for Mock Data}

\subsection{Derived and Implemented Analytic Energy and Momentum Gradient}
We have previously calculated $\nabla E$ and $\nabla L_z$ using numerical derivatives. We can do it better by finding the analytic expression for more accuracy.

First step is to write down the unit vector conversion between cylindrical coordinate and Cartesian coordinate since we work in Cartesian and the galpy potential is in cylindrical.
$$ \hat{R} = \cos(\phi) \hat{x} + \sin(\phi) \hat{y} $$
$$ \hat{\phi} = - \sin(\phi) \hat{x} + \cos(\phi) \hat{y} $$

\subsubsection{Deriving $\nabla L_z$}
The angular momentum per mass in z direction, written as a function of the 6 phase space coordinate, is given by
$$ L_z(x,y,z,v_x,v_y,v_z) = Rv_T $$
where $v_T$ is the tangential velocity in polar coordinate.

To put it more explicitly, since
$$ \phi = \arctan \left( \frac{y}{x} \right) $$
we have
\begin{align*}
\frac{d \phi}{dt} &= \frac{\partial}{\partial x} \left[ \arctan \left( \frac{y}{x} \right) \right] \frac{dx}{dt} + \frac{\partial}{\partial y} \left[ \arctan \left( \frac{y}{x} \right) \right] \frac{dy}{dt} \\
&= \frac{1}{1 + (y/x)^2} (y) (-x^{-2}) v_x + \frac{1}{1 + (y/x)^2} (\frac{1}{x}) v_y \\
&= \left( \frac{1}{1 + (y/x)^2} \right) \left[ \frac{-yv_x}{x^2}+\frac{v_y}{x}
  \right] \\
&= \left( \frac{x^2}{x^2 + y^2} \right) \left[ \frac{-yv_x + xv_y}{x^2} \right] \\
\frac{d \phi}{dt} &= \frac{xv_y - yv_x}{R^2}
\end{align*}

Since $v_T = R\dfrac{d \phi}{dt}$ and $L_z = Rv_T$, we have
$$ L_z = R^2\frac{d \phi}{dt} $$
Substituting,
$$ L_z = xv_y - yv_x $$

Taking the gradient in Cartesian,
$$ \nabla L_z = \frac{\partial L_z}{\partial x} \hat{x} + \frac{\partial L_z}{\partial y} \hat{y} +\frac{\partial L_z}{\partial z} \hat{z} + \frac{\partial L_z}{\partial v_x} \hat{v_x} + \frac{\partial L_z}{\partial v_y} \hat{v_y} + \frac{\partial L_z}{\partial v_z} \hat{v_z} $$
$$ \nabla L_z = v_y \hat{x} - v_x \hat{y} -y \hat{v_x} + x \hat{v_y} $$

In vector component form,
$$ \nabla L_z = [v_y, - v_x , 0, -y, x, 0] $$

\subsubsection{Deriving $\nabla E$}
The energy per mass is given by 
\begin{align*}
E(x,y,z,v_x,v_y,v_z) &= K(v_x,v_y,v_z) + \Phi(x,y,z) \\
&= \frac{1}{2}(v_x^2 + v_y^2 + v_z^2) + \Phi(x,y,z)
\end{align*}
Since $K$ does not depend on position,
\begin{align*}
\nabla K &= \frac{1}{2}(\frac{\partial v_x^2}{\partial v_x} \hat{v_x} +\frac{\partial v_y^2}{\partial v_y}\hat{v_y} + \frac{\partial v_z^2}{\partial v_z} \hat{v_z}) \\
&= v_x \hat{v_x} + v_y \hat{v_y} + v_z \hat{v_z} 
\end{align*}
The gradient of potential energy is negative force. Since galpy already implements this in cylindrical coordinate, we take the cylindrical gradient here. Since the potential is independent of velocity, our gradient only concerns the position.
\begin{align*}
 \nabla \Phi &= \hat{R} \frac{\partial \Phi}{\partial R} + \hat{\Phi} \frac{1}{R}\frac{\partial \Phi}{\partial \phi} + \hat{z}\frac{\partial \Phi}{\partial z} \\
&= - F_R \hat{R}  -F_{\phi} \hat{\Phi} - F_z \hat{z}
\end{align*}
Substituting the polar coordinate unit vector conversion to Cartesian,
\begin{align*}
 \nabla \Phi &= - F_R (\cos(\phi) \hat{x} + \sin(\phi) \hat{y})  -F_{\phi} (- \sin(\phi) \hat{x} + \cos(\phi) \hat{y}) - F_z \hat{z} \\
 &= \hat{x}(F_{\phi} \sin(\phi)- F_R \cos(\phi)) + \hat{y}(- F_R \sin(\phi) -F_{\phi} \cos(\phi)) - F_z \hat{z} \\
\end{align*}

Therefore, the gradient of energy is
$$ \nabla E =  \hat{x}(F_{\phi} \sin(\phi)- F_R \cos(\phi)) + \hat{y}(- F_R \sin(\phi) -F_{\phi} \cos(\phi)) - F_z \hat{z} + v_x \hat{v_x} + v_y \hat{v_y} + v_z \hat{v_z} $$

In vector component form, 
$$ \nabla E = [F_{\phi} \sin(\phi)- F_R \cos(\phi),- F_R \sin(\phi) -F_{\phi} \cos(\phi), - F_z ,v_x, v_y, v_z ] $$

\section{Ayush’s Accomplished Tasks}


\section{Michael's Accomplished Tasks}


\section{Mathew's Accomplished Tasks}


\end{document}