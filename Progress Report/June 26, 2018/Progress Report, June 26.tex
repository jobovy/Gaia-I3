\documentclass[12pt]{article}
\usepackage{amsfonts}
\usepackage{amsmath}
\usepackage{amsthm}
\usepackage{graphicx}
\usepackage{fancyhdr}

\renewcommand{\headrulewidth}{0.4pt}

\addtolength{\oddsidemargin}{-.875in}
\addtolength{\evensidemargin}{-.875in}
\addtolength{\textwidth}{1.75in}
\addtolength{\topmargin}{-.875in}
\addtolength{\textheight}{1.75in}

\begin{document}
\title{Progress Report: (Week of June 26)}
\author{Samuel Wong}
\date{June 26, 2018}
\maketitle


\section{Samuel's Accomplished Tasks}
\subsection{Added Randomized Sign for vR and vz in Sample\_V\_On\_Set}
I realized that the reason fractional error is so large is that vR and vz have a mean of zero due to steady state. By averaging them in the function, it is not right because it decreases he standard deviation. Also, by interpolating, we miss the fact that the numbers are 50\% of the time positive and half the time negative. So I changed Sample\_V\_On\_Set such that it only average vT a few times before putting them into interpolation. vR and vz only get sampled once and their signs get randomnized.
\subsection{Came Up with Using KS Statistics Test For Sample\_V\_On\_Set}
Although fractional error works well with vT if repeated a few times, it is still bad for vR and vz since those are random numbers scattered around 0. So I test the result of vR and vz of Sample\_V\_On\_Set by putting a real vR generated by sampleV and a set of vR generated by interpolation. Then I put those two sets in to a KS statistic test to see whether it can tell they are from the same parent distribution. The higher the p-value, the better because it is harder for KS to tell the difference. We are currently aiming for a p-value of 0.7.
\subsection{Main Program Cluster and Plot}
Combined Michael's Kmeans program with the main program into main\_program\_cluster. Now it takes a large samples, use kmeans to get some representatives (currently 1\%), and then evaluate dot products on them. It then generates a 3 dimensional scatter plot of the maximum absolute value of dot product. So far, trying a few thousands stars with Gaia, it seems the dot product are nowhere near zero. They average at around 0.6.
\subsection{Retested Fake Density}
Due to failure in showing our program can distinguish no I3 with Dehnen df, we suspect either KDE, mock data generation, or main program is wrong.
Using the main program cluster, I tested the fake density again, which is a deliberate density function that has no I3. I plotted the result and to our relief, all of the dot products are at 0. So this proves that the main program works just fine.

\section{Ayush’s Accomplished Tasks}


\section{Michael's Accomplished Tasks}
\subsection{Incorporated STD option}
Wrote a function in tools to apply a variable STD cut to and 6D sample. We are thinking of using a 3 STD cut.

\section{Mathew's Accomplished Tasks}


\end{document}